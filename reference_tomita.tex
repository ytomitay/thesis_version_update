\begin{thebibliography}{9}
  \bibitem{jinkou} 総務省統計局, 
    ``人口推計 (平成30年 (2018年) 6月確定値, 平成30年 (2018年) 11月概算値 (2018年11月20日公表)." \\https://www.stat.go.jp/data/jinsui/new.html  (参照 2018-11-23)
  \bibitem{youkaigo1} 厚生労働省,
    ``介護保健事業状況報告 H30年8月分 結果の概要." \\https://www.mhlw.go.jp/topics/kaigo/osirase/jigyo/m18/dl/1808a.pdf  (参照 2018-11-23 )
   \bibitem{youkaigo2} 厚生労働省,
    ``介護保健事業状況報告 H29年8月分 結果の概要." \\https://www.mhlw.go.jp/topics/kaigo/osirase/jigyo/m17/dl/1708a.pdf
    \bibitem{kaigogenin} 内閣府,
    ``平成30年版高齢社会白書(全体版)第2節 高齢期の暮らしの動向 健康・福祉 (2)65歳以上の者の介護."\\http://www8.cao.go.jp/kourei/whitepaper/w-2018/html/zenbun/s1\_2\_2.html(参照
    2018-11-23)
    \bibitem{danmen} 池添冬芽, 浅川康吉, 島浩人, 市橋則明.
    "加齢による大腿四頭筋の形態的特徴および筋力の変化について-高齢女性と若年女性との比較*- Journal of Computer Aided Chemistry, vol34(2007), p232-238
    \bibitem{elastography}  荒木力(2011), 
    "エラストグラフィ徹底解説-生体の硬さを画像化する", 秀潤社, p.99
    \bibitem{onkyou} J. Greenleaf, 
    “Quantitative cross-sectional imaging of ultrasound parameters”, \\Ultrason. Symp. Proc., 
    pp.989-995, 1977.
    \bibitem{cure1} N. Duric et al.,
     “Detection of breast cancer with ultrasound tomography: First results with the Computed Ultrasound Risk Evaluation (CURE) prototype,” Med. Phys., vol.34, no. 2, pp. 773-785, 2007.
     \bibitem{cure2} C. Li, N. Duric, P. Littrup, and L. Huang, 
     “In vivo Breast Sound-Speed Imaging with Ultrasound Tomography,” Ultarasound Med. Biol., vol. 35, no. 10, pp. 1615-1628. 2009.
     \bibitem{senpai} 中村弘文(2014), 
     "リングアレイを用いた超音波CTにおける屈折を考慮した音速再構成手法の開発", pp. 10-12.
     \bibitem {elastography} 椎名毅,
     "超音波エラストグラフィの研究開発の現状・動向", MEDICAL IMAGING TECHNOLOGY Vol.32, No.2, pp. 63-68, 2014
     \bibitem{siina_2} 椎名毅,
     %エラストグラフィ: 原理 植野 映編
     "実践乳房超音波診断-基本操作, 読影, 最新テクニック-", 中山書店, 東京, 2007.
     \bibitem {onkyouhousya}
     "音響放射圧を用いたイメージング装置の生体への影響について.", 日本超音波医学会, 機器及び安全に関する委員会報告会, 2009. \\http://www.jsum. or.jp/committee/m\_and\_s/acoustic\_radiation.html
     \bibitem{probe} 千原國宏,     
     "超音波", コロナ社(2001)
\end{thebibliography}