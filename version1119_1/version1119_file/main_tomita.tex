%------------------------------------------------------------------------------
\documentclass[master]{cimt_tomita}
% オプションについては,マニュアルを参照.
% \documentclass[master,oneside]{cimt} 

% 必要とするパッケージがあれば,ここで指定する.
\usepackage{graphicx}
\usepackage{url}
\usepackage[fleqn]{amsmath}
\usepackage[psamsfonts]{amssymb}

\usepackage{algorithm}
\usepackage{algorithmic}

\renewcommand{\algorithmicrequire}{\textbf{Input:}}
\renewcommand{\algorithmicensure}{\textbf{Output:}}

\usepackage{layout}
\usepackage{color}

\usepackage{style}

% 論文タイトル
\jtitle{超音波CTを用いた下肢組織の\\動力学的解析手法の開発}
% 長い時には自動的に改行されるが,次のように明示的に改行することもできる.
% \jtitle{長いタイトルを\\このように改行位置を指定して組む}

% 著者名
\jauthor{学籍番号 03-170225\\冨田 結子}

% 指導教員
\supervisor{高木 周 教授}

% 提出月.
\handin{2019}{2}

%学籍番号

\begin{document}

% 表紙と表紙裏
\maketitle 

% ここから前文
\frontmatter

% 概要
\begin{jabstract}
\input jabst.tex
\end{jabstract}

% 目次
\tableofcontents

% ここから本文
\mainmatter

\input 01_introduction_tomita.tex
\input 02_principle_tomita
\input 03_relatedresearch_tomita.tex
\input 04_experiment_tomita.tex

% ここから後付
\backmatter

% 参考文献: BibTeX を使う場合の例 (styleは適宜選択)
\bibliographystyle{junsrt}
\bibliography{references.bib}

% 参考文献: 直接記述する場合の例
%\input biblio.tex

% 謝辞 (前文においても良い)
\begin{acknowledgements}
\input ack.tex
\end{acknowledgements}

%付録 (必要な場合のみ)
\appendix
\input appendix_tomita.tex

\end{document}
