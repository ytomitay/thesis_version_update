本論文は, 筆者が東京大学工学部機械工学科流体工学研究室(高木・杵淵研究室)に在籍中に行った研究についてまとめたものです. この論文を執筆するにあたり, 多くの方々にご協力頂き, まとめあげることができました. ここに末筆ながら感謝の辞を述べさせて頂きます.
\\
\\\ \ \,高木周教授には, 指導教員とを引き受けて頂き, 研究会を始め様々な場面で大変お世話になりました. 研究室でお会いするたびに, 「研究は楽しい?」など私の様子を気にかけてくださり, 心地良い環境で研究をすることができました. 心の底から研究を愛し, 楽しんでいらっしゃるお姿を間近で拝見することができ, 未熟者ではありますが, 研究とは何たるかということを少し知ることができた気がいたします.
\\\ \ \,東隆教授には, 医療班として研究において大変お世話になりました. グループミーティングでは, 的確なご指摘を頂き, 学部4年生での4月では右も左もわからなかった状態から自身の研究テーマに対して自分なりの解釈を述べられるようにまでなりました. お忙しいにも関わらず, 私の拙い質問にも快く答えて頂き大変感謝しております. 
\\\ \ \,杵淵郁也准教授には, 研究生活においてお世話になりました. 研究会や中間試問などで, 自身の研究で何かにつまづいた時にヒントを下さり, 前に進めることがありました. 快適に研究生活を送れたのも氏のお陰であります. 感謝申し上げます. 
\\\ \ \,吉本勇太助教授には, 研究会において, 自身の研究について多くの助言を頂き, 研究だけでなく多くのお仕事をこなしていらっしゃる姿を見て, 感銘を受けました. 
\\\ \ \,東京大学大学院 工学研究科 渡部 嘉気さんには, 同じ医療班として様々なアドバイスを頂きました. 才能に溢れ, いつも鋭いご指摘を頂きとても尊敬しております. 博士課程へのご進学おめでとうございます. ますますのご活躍をお祈り申し上げております. 
\\\ \ \,東京大学大学院 工学研究科 辻本雄太郎さんには, 直属の先輩として卒業論文に取り組むにあたり大変お世話になりました. ご自身の研究でもお忙しい中, 私の研究についても真剣なアドバイスをくださり, 時には一緒に実験や試問用の資料作成にもご協力頂きました. お話も楽しく, いつも優しい辻本さんには本当に感謝申し上げます. 
\\\ \ \,研究室秘書の五十川順子氏には, 研究テーマの集計や日頃の研究生活での身の回りのお世話をして頂きました. 快適な研究生活を送れたのも氏のお陰であります. 
\\\ \ \,修士2年の江口雅大氏は, 研究室でいつも話題の中心でした. 北村進太郎氏は, 研究室を明るい笑顔で和ませてくれました. 小林拓矢氏は, 優しかったです. 祖父江聡士氏は, 鍛えていらっしゃってすごいなと思いました. 田再泓氏は, 研究熱心でいらっしゃいました. 松本浩史氏は, 優れた人間性と研究力で多くの人の信頼を勝ち得ていらっしゃいました. 渡辺力氏は自身のペースをしっかり持っていらっしゃいました. 
\\\ \ \,修士1年の笹岡憲也氏は, 同じ医療班として懇親会などでお世話になりました. 院試などでもアドバイスをくださり感謝しております. 佐藤匠氏は, いつもとても優しく接して頂きました. 杉山颯氏は, いつも面白くて研究室のムードメーカでいらっしゃいました. 辻孝仁氏は, 活発に色々なことをしていらっしゃいました. 藤原裕貴氏は, レク係として多くの企画を運営してくださいました. 予定が合わなかったりで行けなくて本当に申し訳ないです. 堀直樹氏は, とても優秀でいらっしゃいました. 途中からスウェーデンに行ってらっしゃりすごいなと思いました. 村田健吉氏は, ほのぼのとした雰囲気を持っていらっしゃいました. 氏がいらっしゃると研究室が落ち着くような気がします. 屠正月氏は, 途中から研究室にいらっしゃったにも関わらずすぐに研究室に慣れていらっしゃいました. 呉梦雅氏は, とても優しくプレゼントをくださったり, みんなのためにケーキを買ってきてくださったりしました. 
\\\ \ \,同期の周楚凡氏は,  金魚を飼っていたり, 面白い映画を知っていたり研究室の席も隣でたくさん笑わせてもらいました. 菅原稜太氏は, ダーツとボーリングがとても上手で, 研究熱心でありとても尊敬しております. 伊達寛紀氏には, 絶妙な言葉のチョイスでいつも笑わせてもらいました. 山本凌氏は, 笑い声が特徴的で, 彼がいると研究室全体が明るくなりました. 今井宏樹氏は水泳が得意で. 運動も勉強も遊びも何でも楽しんでいる姿にこちらも楽しい気持ちになりました. 上島千拓氏は, 類稀なる研究遂行能力だけではなく, 独特の世界観でとても楽しく会話することができました. 
\\\ \ \,最後に, 私が卒論の執筆において弱音を吐いた時も優しく見守ってくれた家族, 友人など, 全てのお世話になった方々に感謝の意を示させて頂きます. 本当にありがとうございます. 今後ともどうぞよろしくお願い致します. 
\begin{flushright}
  2019年2月1日 冨田結子
\end{flushright}
